%%%%%%%%%%%%%%%%%%%%%%%%%%%%%%%%%%%%%%%%%%%%%%%%%%%%%%%%%%%%%%%%%%%%%%%%%%%%%%%%
% Cover page for project summary
%
% Based on DOE guidance, the cover includes the title and a table listing all
% investigators.  The abstract is separate.
%
% This implementation uses the csvsimple-l3 package to read 
% CONFIG-investigators.csv and display it as a table.  Plenty of tweaking (or
% even wholesale replacement) opportunities if you want something different.
%%%%%%%%%%%%%%%%%%%%%%%%%%%%%%%%%%%%%%%%%%%%%%%%%%%%%%%%%%%%%%%%%%%%%%%%%%%%%%%%

\vfill      %% I don't think this does anything before the chapter heading

\chapter*{\proposaltitle}

\vspace{0.5in}   %% IMHO, fixed vertical spacing looks better than \vfill

% Use csvsimple-l3 to create a table from CONFIG-investigators.csv
%
% The default format displays \rolelpi and \rolecpi rows in bold font on a light
% gray background.  \rolesp is not displayed (makes the others stand out
% more).
%
% Given this formatting, the most logical ordering for the input data is:
% 1. Lead institution followed by additional institutions in alphabetical order
%    a. Institutional PI followed by additional investigators in alpha order
%
% An xcolor specification for a very light gray to highlight lead PIs and co-PIs
\definecolor{verylightgray}{rgb}{0.90,0.90,0.90}
%
% TODO: as written, this generates warnings about misplaced \noalign each
% time the \rowcolor is used.  Not clear why.  Googling has so far been
% unhelpful.
\csvreader[
    head to column names,
    centered tabular = lll,
    table head = \emph{Investigator} & \emph{Institution} & \emph{Role} \\\hline,
    ]{CONFIG-investigators.csv}{}{%
    \ifthenelse{\equal{\role}{\rolelpi} \OR \equal{\role}{\rolecpi}}{%
    \rowcolor{verylightgray}%
    \textbf{\firstname~\lastname} & \textbf{\glsentrylong{\instacro}} & \textbf{\role}
    }{%
    \firstname~\lastname & \glsentrylong{\instacro} &
    }%
}

\vfill